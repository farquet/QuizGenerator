\documentclass[11pt,a4paper]{report}

\usepackage{verbatim}
\usepackage{epsfig}
\usepackage{theorem}
\usepackage{graphicx}
\usepackage{amsmath}
\usepackage{amssymb}
\usepackage[dvips]{color}
\usepackage[dvipdfm,pdftitle={Cryptography and Security - Survey 1},pdfpagemode=None,colorlinks=true,linkcolor=black,urlcolor=blue]{hyperref}
\usepackage{fullpage}
\usepackage{subfigure}
\usepackage{algorithmic}
\usepackage{algorithm}
\usepackage{lib/verbtext}
\usepackage{lib/security}


\input{lib/macros.tex}



% +-[A décommenter pour avoir la solution]--------------+


\solutiontrue



% +-[document]------------------------------------------+
\begin{document}


% +-[header survey]-------------------------------------+
\input{lib/header_fullpage.tex}
\begin{center}
\sf
  \Large{Cryptography and Security}

  \vspace{0.3cm}
  \Large{2012 -- 2013}

  \vspace{0.3cm}
  \Large{Survey n$^{\text{o}}$ 3}

  \vspace{0.5cm}
\end{center}

\hfill{Name:~}{\vrule width 7cm height 0.4pt}

\bigskip

\hfill{SCIPER n$^{\text{o}}$:~}{\vrule width 7cm height 0.4pt}

\begin{comment}
 \question[]{}
 {}
 {}
 {}
 {}
\end{comment}

\question[1]{Tick the \textbf{asymmetric} primitives.}
 {Digital Signatures}
 {MACs}
 {Stream Ciphers}
 {Block Ciphers}

%\question[1]{In AES, changing the order of which of the following two operations do not change the result?}
\question[1]{Consider the following pairs of AES operations. In which, the order
	of the two operations does \emph{not} matter?}
 {SubBytes-ShiftRows}
 {AddRoundKey-MixColumn}
 {SubBytes-MixColumn}
 {SubBytes-AddRoundKey}

 
 
 \question[3]{Tick the \textbf{false} statement regarding the DES round function.}
 {There is an expansion operation $E$ from 32 to 48 bits.}
{A round key is XORed to an internal register.}
{There are $8$ identical S-boxes (substitution boxes) of size $6 \times 4$.}
 {There is a permutation $P$ on 32-bits.}
 
\question[2]{Which mode of operation is similar to a stream cipher?}
 {ECB}
 {OFB}
 {CFB}
 {CBC}

  \question[2]{Ciphertext stealing ... }
 {results in the expansion  of the plaintext.}
 {prevents the expansion of the ciphertext.}
 {requires an additional secret key.}
 {steals ciphertext which results in lack of encrypted data.}


 %	\question[4]{CBC-MAC}
%	{is a block cipher.}
%	{is a good way of designing a MAC from an encryption scheme.}
%	{should never be used as an internal component of a system.}
%	{is a bad MAC by itself.}
 \question[4] {Tick the \emph{false} assertion.}
 {The Rijndael design contains S-boxes.}
 {The Rijndael design uses $\text{GF}(2^8)$ arithmetics.}
 {The Rijndael design contains a linear layer --- ShiftRows and MixColumn.}
 {The Rijndael design is based on one-way functions.}

	\question[3]{CCM}
	{is a block cipher.}
	{is a hash function.}
	{is an authenticated mode of operation.}
	{stands for coded ciphertext mode.}

	\question[2]{Tick the \emph{false} assertion.}
	{For MACs, in the chosen message attack model, the adversary can force the
	sender to authenticate a selected message.}
	{For MACs, the goal of an adversary is to decrypt a message.}
	{For MACs, in the known message attack model, the adversary can only read
	authenticated messages in transit.}
	{For some MACs, the adversary can forge authenticated messages without recovering the secret
	key.}

%	\question[4]{Tick the \emph{false} assertion.}
%	{A PRNG can be obtained from a stream cipher.}
%	{A PRNG can be obtained from a block cipher.}
%	{Some versions of SSL were using a bad PRNG.}
%	{A PRNG requires a secret key.}

	\question[4]{Tick the \emph{correct} assertion.}
	{DES uses 256-bit keys.}
	{AES uses 80-bit keys.}
	{MD5 has a 160-bit digest.} 
	{SHA-1 has a 160-bit digest.}

 \question[1]{Tick the \emph{false} assertion.}
 {$\epsilon$-universal hash functions can be used universally, i.e., in any
 cryptographic application.}
 {$\epsilon$-universal hash functions are families of functions.}
 {An $\epsilon$-XOR-universal hash function $h_k$, for given $x,y$, $x \neq
	 y$ satisfies that for any $a$
	 %\begin{equation*}
 $\Pr[h_k(x)\oplus h_k(y) = a] \leq \epsilon$
 %\end{equation*}
 over a random $k$.}
 {$\epsilon$-universal hash functions are used in WC-MAC.}
\end{document}
